\documentclass[a4paper,11pt]{article}

\usepackage{amsmath}
\usepackage{booktabs}
\usepackage{multirow}

\usepackage{amsthm}
\newtheorem{mydefinition}{Definition}
\newtheorem{lemma}{Lemma} 

\title{ISGS \LaTeX Beginner Workshop 2016\\
\LaTeX{} Exercise Sheet \#2}
\author{
Patrick Wolf \\ patrick.wolf@cs.uni-kl.de}
\date{December 02, 2016}


\begin{document}
\maketitle

\tableofcontents

\section{Booktabs with Multicolumn \& Multirow}
\begin{tabular}{cccc}
\toprule
1&2&\multicolumn{2}{c}{3}\\
\multirow{2}{*}{4}&5&6&7\\
&8& \multicolumn{2}{c}{\multirow{2}{*}{9}}\\
10&11&&\\
\bottomrule
\end{tabular}

\section{Math}
\subsection{Alignment}
\begin{align}	% the align environment automatically aligns equations at the = mark
	a_{1} x  + a_{2} x^{2} = 0\\
	b_{1} x + b_{2} x^{2} + b_{3} x ^{3} =0
\end{align}

\subsection{Dot-less Vectors}
\begin{equation*}
	\vec{k} = \vec{\imath} \times \vec{\jmath}
\end{equation*}

\subsection{Golden Ratio}	% note that math mode can be nested within the emphasis
Two quantities $a$ and $b$ are said to be in the \emph{golden ratio $\varphi$} if
\begin{equation}
\frac{a}{b} = \frac{a+b}{a} = \varphi
\end{equation}
By the way: $\varphi = 1+\frac{1}{\varphi}$.


\subsection{1+1=2}
Often you see complicated equations like this:
\begin{equation} \label{eq:1+1}
1 + 1 = 2
\end{equation}
This complicated formula can be significantly simplified. As you know,
\begin{equation} \label{eq:ln}
1 = \ln e
\end{equation}

\begin{equation}
1 = \sin^2 \alpha + \cos^2 \alpha
\end{equation}
and
\begin{equation} \label{eq:sum}
2 = \sum_{n=0}^\infty \frac{1}{2^n}
\end{equation}
Inserting equations \eqref{eq:ln} -- \eqref{eq:sum} into \eqref{eq:1+1} yields
\begin{equation}
\ln e + (\sin^2 \alpha + \cos^2 \alpha) = \sum_{n=0}^\infty \frac{1}{2^n}
\end{equation}
Furthermore,
\begin{equation}
1 = \cosh \varphi \cdot \sqrt{1 - \tanh^2 \varphi}
\end{equation}
and
\begin{equation}
e = \lim_{c \rightarrow \infty} \left[1 + \frac{1}{c} \right]^c
\end{equation}
With this knowledge we finally can simplify equation~\eqref{eq:1+1} to
\begin{equation}
\ln \left(\lim_{c \rightarrow \infty} \left[1 + \frac{1}{c} \right]^c\right) + (\sin^2 \alpha + \cos^2 \alpha) = \sum_{n=0}^\infty \frac{\cosh \varphi \cdot \sqrt{1 - \tanh^2 \varphi}}{2^n}
\end{equation}

\section{Theorems \& Definitions}

\begin{mydefinition}
   This is a user definition definition.
   \label{def_1}
\end{mydefinition}

\begin{lemma}
   This is a lemma. Have a look at the number of Definition~\ref{def_1} and this lemma.
\end{lemma}


\end{document}