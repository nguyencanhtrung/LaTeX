\documentclass[a4paper, 10pt]{article}
% preample
\usepackage{hyperref}    %embedded email, automatically link \ref to section
\usepackage{url}				%embedded URL
\usepackage{todonotes} %todo notes
\usepackage{booktabs}  %tabular package
\usepackage[fleqn]{amsmath} 	%align formular
\usepackage[linesnumbered]{algorithm2e} % algorithm environment
\usepackage[utf8]{inputenc}
\usepackage{makeidx}
\makeindex

% define new commands
\newcommand{\smileyitem}{:-)}

% opening section
\title{ISGS \LaTeX { }Beginner's Workshop 
		\\ \LaTeX 2006 - Exercise Sheet \# 1 }
\author{Trung C. Nguyen 
			\\ email \href{mailto:nguyencanhtrung@me.com}{nguyencanhtrung@me.com} }

% Main document
\begin{document}
	\maketitle
	\tableofcontents
	
	\newpage
	\listoftodos
	
	\newpage
	\todo{Adding Task here}
	
	%{\raggedright\Large\textbf{Task}}\\
	\section*{Task}  % it won't be included to table content\\
	Re-program the entire exercise sheet and have a look at \cite{wikibook} if you need any help solving the problems at home
	
	\paragraph{Task} \hspace{0pt} \\           % adding \hspace to be able do line break here
	Re-program the entire exercise sheet and have a look at \cite{wikibook} if you need any help solving the problems at home
	%-----------------------------------------------------------
	% First SECTION
	%-----------------------------------------------------------
	\section{Nested Lists}
	This is a list which should be named in the index.
		\begin{itemize}
			\item Red
			\item Green
				\begin{enumerate}
					\item (striped)
					\item (dotted) \linebreak
					 		  no number
					\item[7] different number
				\end{enumerate}
			\item White\footnote{note all the white space on this page}
		\end{itemize}
	In section ~\ref{sec:rclltab}, we continue with tables		
	
	\newpage
	%-----------------------------------------------------------
	% Second SECTION
	%-----------------------------------------------------------
	\section[RLLC-Tabular]{RLLC-tabular}
	\label{sec:rclltab}
	The section title does not match the table of contents\dots \\
		\indent Here is a gap of 5\,cm.	% \,  to get a gap of half space
		\vspace{5cm}								% vertical gap of 5 cm
		
		\begin{table}[h!]
			\begin{tabular}{rcll}
			Color&Fruit&Color&Clothing\\
			Red&Pear&White&Shirt\\
			Green&Pear&White&Shirt\\
			Yellow&Banana&Blue&Hat
			\end{tabular}
		\end{table}
	
		\begin{table}[h!]
			\caption{The table uses the \texttt{booktabs} package for the rules. 
							Additionally, a \texttt{table} environment encapsulates a \texttt{tabular} environment. 
							The \underline{caption} uses the \texttt{caption} command.}
			\label{table 1}
			\begin{tabular}{c p{0.7\textwidth}}
				\toprule
				\textbf{Centered}  & \textbf{Text} \\
				\midrule
				Centered     & Long text. Long text. Long text. Long text. Long text. Long text. 
									Long text. Long text. Long text. Long text. Long text. Long text. 
									Long text. Long text. Long text. Long text. Long text. \\
				Missing text & 			\\ 
				\bottomrule
			\end{tabular}
		\end{table}

		
	A reference to the \ref{table 1} on page \pageref{table 1}.
	
	%-----------------------------------------------------------
	% Third SECTION
	%-----------------------------------------------------------
	\section{Advanced \lq\lq Math\rq\rq}
		\subsection{Dollars \& Pounds}{\label{section 3.1}}
			\todo{Left alignment not done}
			\begin{flalign}
				\$\,1 + \pounds\,1 &=  \,? \\
				\$\,1 - \pounds\,1 &=  \,?
			\end{flalign}
			

		
		\subsection{Lengths}
			23\,mm + 0.004\,km =?
		\subsection{Maths for Biologists\index{biology}}
		How much wood would a woodchuck\index{biology!woodchuck} chuck if a woodchuck\index{biology!woodchuck} would chuck wood?
		
	%-----------------------------------------------------------
	% Fourth SECTION
	%-----------------------------------------------------------
	\section{Manual Index: Using References}
	We can surely say that section \ref{section 3.1} is on page \pageref{section 3.1}
	%-----------------------------------------------------------
	% Fifth SECTION
	%-----------------------------------------------------------
	\section{How to put a Giraffe into a Refrigerator}
	We have found a simple solution for this age-old problem. It can best be described using the \texttt{algorithm} environment which we get using
	\begin{verbatim}
	\usepackage{algorithm2e}
	\end{verbatim}
	in the preamble

	\verb|\usepackage{algorithm2e}| 	% similar meaning as \begin{verbatim} 
	
	\begin{algorithm}
	Open Door\;
	\While{ Giraffe not Inside}{
			\eIf{Still power} 
				{Push\;}
				{Rest some time \;}
	}
	Close Door\;
	\caption{Putting a Giraffe into the Fridge}
	\label{algorithm 1}
	\end{algorithm}
	
	Another reference, it points to algorithm~\ref{algorithm 1}. Did you notice that the numbers adopt to the content?
	
	%-----------------------------------------------------------
	% Sixth SECTION
	%-----------------------------------------------------------
	\section{Emphasizing Things}
	With the command \texttt{\textbackslash emph{\dots}} you can \emph{emphasize} a word. Its appearance depends--of course--on the currently selected document class and style. \textsl{If the text itself was slanted,} the emphasized words \textsl{would appear upright!}
		\subsection{Some Fonts\index{font}}
		\hspace{3pt} {\todo{adding indent here}}There are {\huge huge} and {\tiny tiny} fonts. For instance, text may be \textbf{bold}\index{font!bold|textbf} or \textit{italic}\index{font!italic|textit}. It might also have another font family like Roman Family or {\sffamily Sans Serif Family}.
	
	%-----------------------------------------------------------
	% Seventh SECTION
	%-----------------------------------------------------------
	\section{Quotes}
	This is a quotation.
	\begin{quotation}
	LaTeX (pronounced either Lah-tech or Lay-tech) is a macro package based on TeX created by Leslie Lamport~[\cite{wikibook_1}, page~\pageref{ref}].
	\end{quotation}
	
	{\todo{Removing indent here}}Have a look at the bibligraphy. There seems to be an item which was not cited within this document.
	
	%-----------------------------------------------------------
	% Eighth SECTION
	%-----------------------------------------------------------
	\section{Commands}
	A new defined command \texttt{\textbackslash smileyitem} generates a list\index{list|textbf} entry with a smiley icon
	
	\paragraph*{Example} \hspace{0pt} \\
	The command \texttt{\textbackslash smileyitem} used within an \texttt{itemize} or \texttt{enumerate} environment looks like:
	\begin{enumerate}
		\item[\smileyitem] This is a smiley item \dots
	\end{enumerate}
	
	\begin{itemize}
		\item  and this not 
	\end{itemize}
	
	%-----------------------------------------------------------
	% Nineth SECTION
	%-----------------------------------------------------------	
	
	\medskip
	
	% Adding bib
	\bibliography{bibfile} {\label{ref}}
	\bibliographystyle{ieeetr}
	\todo{Incomplete reference, not cite -> not appear in the reference; How to put URL on ref}
	
	\newpage
	\printindex
\end{document}
