\documentclass[a4paper,10pt]{article}

\usepackage[linesnumbered]{algorithm2e}
\usepackage{makeidx}
\usepackage{url}
\makeindex

\usepackage{booktabs}

\newcommand{\smileyitem}
{
 \item[:-)]
}

\title{ISGS \LaTeX~Beginner's Workshop \\\LaTeX{} 2016 -- Exercise Sheet \#1}
\author{Patrick Wolf \\
patrick.wolf@informatik.uni-kl.de}

\date{\today}


\begin{document}
\maketitle

\tableofcontents

\newpage

\section*{Task}
Re-program the entire exercise sheet and have a look at \cite{wikibooks.org-Latex} if you need any help solving the problems at home.

\section{Nested Lists}

This is a list\index{list} which should be named in the index\index{index|textbf}.
\begin{itemize}
	\item Red
	\item Green
		\begin{enumerate}
			\item (striped)
			\item (dotted)
			\item[] no number
			\item[7.] different number
		\end{enumerate}
	\item White\footnote{note all the white space on this page}
\end{itemize}

In section~\ref{sec:rclltab}, we continue with tables.

\clearpage
\section[RLLC-Tabular]{RCLL-Tabular}
\label{sec:rclltab}

The section title does not match the table of contents\dots

Here is a gap of 5\,cm.
\vspace{5cm}

\begin{tabular}{rcll}
Color & Fruit & Color & Clothing\\[2mm]
Red & Apple & Yellow & Boots\\
Green & Pear & White & Shirt\\
Yellow & Banana & Blue & Hat\\
\end{tabular}

\begin{table}[htb]
\caption{The table uses the \texttt{booktabs} package for the rules. Additionally, a \texttt{table} environment encapsulates a \texttt{tabular} environment. The \underline{caption} uses the \texttt{caption} command.}
\label{table}
 \begin{tabular}{c p{0.75\textwidth}}
 \toprule
  \textbf{Centered} & \textbf{Text}\\ \midrule
  Centered & Long text. Long text. Long text. Long text. Long text. Long text. Long text. Long text. Long text. Long text. Long text. Long text. Long text. Long text. \\
  Missing text & \\
  \bottomrule
 \end{tabular}
\end{table}

A reference to the table\index{table} \ref{table} on page \pageref{table}.


%\clearpage
\section{Advanced \lq\lq Math\rq\rq}
\subsection{Dollars \& Pounds}
\label{sec:dollarspounds}
\$\,1 + \pounds\,1 = ?\\
\pounds\,2 $-$ \$\,1 = ?
\subsection{Lengths}
23\,mm + 0.004\,km = ?
\subsection{Maths for Biologists}
How much wood would a woodchuck\index{biology}\index{biology!woodchuck} chuck if a woodchuck would chuck wood?

\section{Manual Index: Using References}
We can surely say that section~\ref{sec:dollarspounds} is on page~\pageref{sec:dollarspounds}.

\section{How to put a Giraffe into a Refrigerator}
We have found a simple solution for this age-old problem. It can best be described using the \verb=algorithm= environment which we get using
\begin{verbatim}
\usepackage{algorithm2e}
\end{verbatim}
in the preamble.
\begin{algorithm}
Open Door\;
\While{Giraffe not Inside}{
\eIf{Still power}{Push\;}{Rest some time;}
}
Close Door\;
\caption{Putting a Giraffe into the Fridge}
\label{algorithm}
\end{algorithm}

Another reference, it points to algorithm~\ref{algorithm}. Did you notice that the numbers adopt to the content?

\section{Emphasizing Things}
With the command \verb|\emph{...}| you can \emph{emphasize} a word. Its appearance depends---of course---on the currently selected document class and style. \textsl{If the text itself was slanted, the \emph{emphasized words} would appear upright!}

\subsection{Some Fonts}
~\index{fonts}
\index{fonts!italic|textit}
\index{fonts!bold|textbf}
\index{fonts!small caps}
There are {\Huge huge} and {\tiny tiny} fonts. For instance, text may be \textbf{bold} or \textit{italic}. It might also have another font family like {\rmfamily Roman Family} or {\sffamily Sans Serif Family}.

\section{Quotes}
This is a quotation.
\begin{quotation}
 LaTeX (pronounced either ”Lah-tech” or ”Lay-tech”) is a macro package based on TeX
created by Leslie Lamport \cite[page 5]{latex}.
\end{quotation}
Have a look at the bibliography. There seems to be an item which was not cited within this document. \nocite{*}

\section{Commands}
A new defined command \verb=\smileyitem= generates a list entry with a smiley icon.

\subsection*{Example}
The command \verb=\smileyitem= used within an \texttt{itemize} or \texttt{enumerate} environment looks like:

\begin{itemize}
 \smileyitem This is a smiley item \dots
 \item and this not
\end{itemize}


\bibliographystyle{plain}
\bibliography{exercise-bibliography}

%\begin{minipage}{0.99\textwidth}
 \printindex
%\end{minipage}

\end{document}